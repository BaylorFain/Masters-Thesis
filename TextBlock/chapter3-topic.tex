
\section{Results}

\subsection{Model simulation}

Using the influenza parameters of Table \ref{tab_params}, infections initiated with 1001365 cells in a dish, an MOI of $10^{-4}$, and no initial virus were simulated. 

For a closer look at the plaques, figure \ref{fig_ZoominDish} is a zoomed in view of the infection at hours 6.5, 11.5, and 16.5. 

\subsection{Convergence Testing}

Three scenarios were examined when testing the convergence of the model: an infection initiated with $10013$ cells in the eclipse phase (Initial Cell); an infection initiated with $10^{12}$ virions (Initial Virus); and a scenario with no infection, but $10^{12}$ virions (Only Virus), examining viral spread and decay only. 

In figure \ref{fig_AspectGraphs} the different viral titer curves are explored further by plotting the measurable characteristics mentioned in section \ref{Viral_transmission} for each time step. 

\subsection{Fitting the model to data}

The model is fit to experimental \emph{in vitro} data \citep{pinilla12} via minimization of the SSR. 

%%%%%%%%%%%%%%%%%%%%%%%%%%%%%%%%%%%%%%%%%%%%%%%%%%%%%%%%%%%%%%%%%%%%%%%%%%%



