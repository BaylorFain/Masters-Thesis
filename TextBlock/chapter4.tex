In the previous chapter the model as applied to SARS-CoV-2 data and shown how the model can be used to study the virus. In this chapter, the findings, extensions of the model, and future work of this thesis will be discussed. I will discuss how the proposed model forms a foundation for future work, how the model will be extended with more cellular processes, and how the application of the model gives insight to disease severity.

old-paper
In the previous chapter an accurate and quick model being applied to experimental data was shown. In the chapter, the findings, limitations, and future work of this thesis will be discussed. I will discuss how the faster speed of the simulations allows for the model to be compared with common practices in the field of computation virology, how the current limitation of a lack of all the cell processes can be addressed, and how the model will be applied in the future.

\section{Findings}

This paper shows that the use of GPUs to accelerate computation of agent-based and partial-differential equation hybrid models allows for simulation results within hours, but with the necessary level of detail to capture individual cell effects, and allows for parameterizing the model quickly. The model in this work accurately replicates the diffusion of a virus, the stages of infection of individual cells, and can be fit to data within hours. While still lacking some of the biology needed for replication of \emph{in vivo} infections, the speed of computation leaves room for incorporation of additional features. Thus, this model implementation forms the foundation of a modeling and simulation tool that can accurately predict in-host viral dynamics and be quickly deployed to help combat the next pandemic. 

\section{Model extensions}

Although our model currently only incorporates cell-free transmission, since the ABM models interactions of each cell in a culture dish, the spatial aspects of different viral transmission routes can be explored in detail. There has been recent interest in viruses that transmit via cell to cell transmission, with ODE \citep{allen15,komarova13,iwami15}, stochastic \citep{graw15}, and ABM \citep{kumberger18,blahut21} models developed to study how cell to cell transmission alters infection dynamics. There are also viruses that cause cells that form syncytia, which are cells that have fused into a single multi-nucleated cell. Not much is known about how syncytia alter infection dynamics, with a recent ODE model attempting to assess the effect of syncytia on viral time course \citep{jessie21}, but spatial effects really need to be included for a proper assessment of the role of syncytia. Finally, advection can be added to the diffusion of the virus particles to more closely mimic the respiratory tract. Recent PDE \citep{quirouette20} and ODE \citep{gonzalez19} models both indicate that the addition of advection can limit the spread of respiratory viruses towards the lower respiratory tract, but the stochasticity included in an ABM might affect this result. 

While the model is able to replicate a typical viral time course during an infection, it is missing many components that play important roles in the infection. For example, the immune response of the host has not been added to the model. The immune response is a large, if not the main, contributing factor to symptoms experienced during a viral infection \citep{manchanda14,zheng18}, but also limits spread of infection itself \citep{dobrovolny13}. ABMs are already used to model various aspects of the immune response \citep{whitman20,kerepesi19,levin16}, so the immune response can be incorporated into the existing ABM/PDM framework. Cell tropism, the preference of virus for one cell type over another, is another feature of viral infections that can be incorporated into the ABM. ODE modeling indicates that cell tropism can lead to longer lasting infections \citep{dobrovolny10}, but will also likely affect the spatial dynamics of infection. Finally, variation in production of virus by individual cells \citep{timm12} can be incorporated to determine how this type of cell heterogeneity affects spatiotemporal infection dynamics.

%%%%%%%%%%%%%%%%%%%%%%%%%%%%%%%%%%%%%%%%%%%%%%%%%%%%%%%%%%%%%%%%%%%%%%%%%%%%%%%%%%%%%

\section{Future Work}

A goal of this research is to not only be able to predict viral infection, but it is to find ways to uncover potential causes of disease severity. The novel coronavirus, SARS-CoV-2, originated in Wuhan, China in late 2019 and rapidly spread around the world \citep{chen20,wu20}. This virus causes the Covid-19 disease which can lead to severe illness needing long hospitalization \citep{sun20,goyal20,jiang20}, but at the same time a significant fraction of those who contract the virus experience an asymptomatic Covid-19 disease \citep{he20}. It is still not entirely clear who is at risk for developing severe disease, although correlations of disease severity with levels of vitamin D \citep{ilie20}, levels of various immune components \citep{liu20imm,liu20imm2,zhang20imm,yang20imm}, and age \citep{borghesi20,zhang20imm} have been noted. There has also been investigation of the possibility of disease severity being linked to initial viral inoculum \citep{little20, guallar20, ghandi20}.

The difference in viral inoculum between patients could be caused by varying amounts of virus in airbone droplets. The major route of transmission for SARS-CoV-2 is by airborne droplets \citep{morawska20}. One study indicates that sneezing and coughing creates a turbulent gas cloud that can cause viral-laden droplets to spread up to 27 feet (\numrange[range-phrase = --]{7}{8}\si{\meter}) \citep{bourouiba20}, and allows the virus to get into the ventilation system of a building. A review of literature on droplet and airborne viral spread concludes that 8 of 10 studies showed that droplets spread further than the 6 foot \citep{bahl20} social distancing recommendation. While personal protective equipment is helpful in reducing the ability of virus to enter the respiratory tract, it is not perfect \citep{mittal20}. All of these factors lead to exposures to vastly different quantities of virus when people are going about their daily activities. Thus it is important to understand whether different initial inocula lead to different viral dynamics in patients. 

There is some evidence from other respiratory viruses that the size of the initial inoculum could play a role in the severity of the illness. An influenza epidemiological modeling study suggests that a higher initial dose can lead to a higher mortality rate \citep{paulo10}. This is corroborated by an influenza in-host modeling study that also finds a correlation between the initial viral dose and survival rate \citep{price15}. Other modeling studies have found dependence of other measures of infection severity on initial dose for influenza \citep{moore20}, respiratory syncytial virus \citep{wethington19}, adenovirus \citep{li14}, and porcine reproductive and respiratory virus \citep{go19}. There are also experimental studies that find a link between dose and infection severity. Experiments using influenza have found inoculum dose dependence of total number of infected cells and area under the curve \citep{manicassamy10}, peak viral titer \citep{ginsberg52,iida63,ottolini05}, viral growth rate \citep{ginsberg52}, and time of viral peak \citep{iida63,ginsberg52}. Experiments with other viruses, such as adenovirus \citep{prince93}, and parainfluenza \citep{ottolini96}, have also shown correlations between initial inoculum and various measures of disease severity. If SARS-CoV-2 shows a similar pattern, initial inoculum should be considered as a possible contributor to infection severity and adverse outcomes.

With this model different scenarios can be tested to help narrow down and rule out causes a different viral infection severities. 

\section{Conclusion}

