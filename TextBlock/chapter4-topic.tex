
\section{Findings}

This paper shows that the use of GPUs to accelerate computation of agent-based and partial-differential equation hybrid models allows for simulation results within hours, but with the necessary level of detail to capture individual cell effects, and allows for parameterizing the model quickly. 

\section{Model extensions}

Although our model currently only incorporates cell-free transmission, since the ABM models interactions of each cell in a culture dish, the spatial aspects of different viral transmission routes can be explored in detail. 

While the model is able to replicate a typical viral time course during an infection, it is missing many components that play important roles in the infection. 

%%%%%%%%%%%%%%%%%%%%%%%%%%%%%%%%%%%%%%%%%%%%%%%%%%%%%%%%%%%%%%%%%%%%%%%%%%%%%%%%%%%%%

\section{Future Work}

A goal of this research is to not only be able to predict viral infection, but it is to find ways to uncover potential causes of disease severity. 

The difference in viral inoculum between patients could be caused by varying amounts of virus in airbone droplets. 

There is some evidence from other respiratory viruses that the size of the initial inoculum could play a role in the severity of the illness. 

With this model different scenarios can be tested to help narrow down and rule out causes a different viral infection severities. 

\section{Conclusion}

