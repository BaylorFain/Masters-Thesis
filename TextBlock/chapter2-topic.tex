
\section{Mathematical Model}

In this work, a two dimensional biological system is simulated with a mathematical model. 

\subsection{Viral transmission} \label{Viral_transmission}

When a virus is spreading among the cells in a culture dish, there is a probability that a healthy cell becomes infected by virus that is not within a cell, but flowing around and above the cell. 

As the viral infection progresses the total amount of virus, in the culture dish, changes and the shape of the total amount of virus over time can change depending on the virus being use for the infection. 

\subsection{Spatial accounting} \label{Spatial_accounting}

To allow for the two dimensional aspect of the culture dish to be represented in the model, the cells are approximated as hexagons. 

\subsection{Agent-based model} \label{ABM}

In an ABM, a system is broken down into smaller units called ``agents''. 

In this work, an ABM governs the transitions a cell makes through the stages of infection: healthy, eclipse, infected, and dead. 

The ABM uses four time arrays to track and transition the cells to different states after infection. 

\subsection{Partial differential equation model} \label{PDM}

PDMs are used to model multiple dimensions; in this work a PDE in hexagonal coordinates is used to model the two-dimensional spatial spread of virus over cells in a culture dish. 

\subsection{Parameters of viral spread}

The eight parameters $\beta$, $\tau_E$, $\eta_E$, $\tau_I$, $\eta_I$, $p$, $c$, and $D$ affect the dynamics of virus spread in the model. 

\subsection{Implementation on GPUs}

As the model becomes more complex, GPU acceleration via parallel programming is used to decrease the simulation run times and therefore increase the number of studies that can be conducted in a given time. 

%%%%%%%%%%%%%%%%%%%%%%%%%%%%%%%%%%%%%%%%%%%%%%%%%%%%%%%%%%%%%%%%%%%%%%%%%%%%%%%%%%
\subsection{Convergence Testing}

Partial differential equations (PDEs) are a popular way to model systems that evolve over both space and time, but often require computers to produce solutions. 

Depending on the choice of numerical scheme, a conditional relationship between $\Delta t$, $\Delta x$, and $\Delta y$ must be met. 

To ensure the simulations are not using more resources than necessary, the space and time discretizations: $\Delta t$, $\Delta x$, and $\Delta y$ need to be optimized. 

\subsection{Data Fitting} \label{Data_Fitting}

As part of our model validation, the model is shown to reproduce viral titer curves observed experimentally. 

To determine the best fit of the model to the experimental data, the sum of square residuals (SSR) is minimized, $$\mathrm{SSR} = \sum_{i=1}^{n} (y_i - \hat y_i)^{2},$$ where $y_i$ is from the experimental data set and $\hat y_i$ is from the simulated data set. 
%%%%%%%%%%%%%%%%%%%%%%%%%%%%%%%%%%%%%%%%%%%%%%%%%%%%%%%%%%%%%%%%%%%%%%%%%%%%%%%%%%%

