For many years, infectious disease modelers have used agent-based models to study the spread of viruses, but the models were too computationally intensive to fully replicate even in vitro experiments. Now, with technological advancements and accessible software, agent-based models can be used to their full potential. This thesis shows an agent-based model that expresses viral transmission and diffusion, can manipulate and track individual cells, and can be fit to real experimental data in a timely manner due to acceleration of computation with graphics processing units (GPUs). The use of GPUs allows simulations to run on desktop computers in a few seconds or minutes, while still simulating an accurate number of cells to replicate  \emph{in vitro} viral infection experiments. This model can now be used to study in-host infections quickly, so that in the event of an outbreak or epidemic a treatment plan and course of action can be developed in less time.
